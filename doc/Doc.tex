\def\allfiles{}
\documentclass[11pt]{article}
\usepackage[a4paper]{geometry}
\geometry{left=2.0cm,right=2.0cm,top=2.5cm,bottom=2.5cm}
\usepackage{ctex} % 支持中文的LaTeX宏包
\usepackage{amsmath,amsfonts,graphicx,subfigure,amssymb,bm,amsthm,mathrsfs,mathtools,breqn} % 数学公式和符号的宏包集合
\usepackage{algorithm,algorithmicx} % 算法和伪代码的宏包
\usepackage[noend]{algpseudocode} % 算法和伪代码的宏包
\usepackage{fancyhdr} % 自定义页眉页脚的宏包
\usepackage[framemethod=TikZ]{mdframed} % 创建带边框的框架的宏包
\usepackage{fontspec} % 字体设置的宏包
\usepackage{adjustbox} % 调整盒子大小的宏包
\usepackage{fontsize} % 设置字体大小的宏包ddddddddd
\usepackage{tikz,xcolor} % 绘制图形和使用颜色的宏包
\usepackage{multicol} % 多栏排版的宏包
\usepackage{multirow} % 表格中合并单元格的宏包
\usepackage{pdfpages} % 插入PDF文件的宏包
\RequirePackage{listings} % 在文档中插入源代码的宏包
\RequirePackage{xcolor} % 定义和使用颜色的宏包
\usepackage{wrapfig} % 文字绕排图片的宏包
\usepackage{bigstrut,multirow,rotating} % 支持在表格中使用特殊命令的宏包
\usepackage{booktabs} % 创建美观的表格的宏包
\usepackage{circuitikz} % 绘制电路图的宏包
\usepackage{float} %
\usepackage{multirow}
\usepackage{indentfirst}
\usepackage{titlesec} %section宏
\usepackage{listings}
\usepackage{paralist}%压缩列表和行间列表环境
\usepackage[perpage]{footmisc}
\usepackage[colorlinks=false]{hyperref}
\usepackage{setspace}
%\usepackage[subfigure]{tocloft}      %必须这么写,否则会报错
%\renewcommand{\cftchapleader}{\cftdotfill{0.6}} %设置chapter条目的引导点间距
%\renewcommand{\cftsecleader}{\cftdotfill{0.6}}
%\renewcommand{\cftsubsecleader}{\cftdotfill{0.6}}
%\renewcommand{\cftchapfont}{\hts}    %设置chapter条目的字体
%\renewcommand{\cftsecfont}{\stxs}    %设置section条目的字体
%\renewcommand{\cftsubsecfont}{\stxs} %设置subsection条目的字体





\titleformat*{\section}{\LARGE\bfseries\songti}
\titleformat*{\subsection}{\Large\songti\bfseries}
\titleformat*{\subsubsection}{\large\songti\bfseries}
\titleformat*{\paragraph}{\large\songti\bfseries}
\titleformat*{\subparagraph}{\large\songti\bfseries}



\definecolor{dkgreen}{rgb}{0,0.6,0}
\definecolor{gray}{rgb}{0.5,0.5,0.5}
\definecolor{mauve}{rgb}{0.58,0,0.82}
\lstset{
	frame=tb,
	aboveskip=3mm,
	belowskip=3mm,
	showstringspaces=false,
	columns=flexible,
	framerule=1pt,
	rulecolor=\color{gray!35},
	%backgroundcolor=\color{gray\Delta 5},
	basicstyle={\small\ttfamily},
	numbers=none,
	numberstyle=\tiny\color{gray},
	keywordstyle=\color{blue},
	commentstyle=\color{dkgreen},
	stringstyle=\color{mauve},
	breaklines=true,
	breakatwhitespace=true,
	tabsize=3,
}


\lstdefinestyle{verilog}{
	columns=fixed,       
	numbers=left,                                        % 在左侧显示行号
	numberstyle=\tiny\color{gray},                       % 设定行号格式
	frame=single,                                          % 不显示背景边框
	backgroundcolor=\color[RGB]{255,255,255},            % 设定背景颜色
	keywordstyle=\color[RGB]{40,40,255},                 % 设定关键字颜色
	numberstyle=\footnotesize\color{darkgray},           
	commentstyle=\it\color[RGB]{0,96,96},                % 设置代码注释的格式
	stringstyle=\rmfamily\slshape\color[RGB]{128,0,0},   % 设置字符串格式
	showstringspaces=false,                              % 不显示字符串中的空格
	language=verilog,                                        % 设置语言
}

\lstdefinestyle{c}{
	columns=fixed,       
	numbers=left,                                        % 在左侧显示行号
	numberstyle=\tiny\color{gray},                       % 设定行号格式
	frame=single,                                          % 不显示背景边框
	backgroundcolor=\color[RGB]{255,255,255},            % 设定背景颜色
	keywordstyle=\color[RGB]{40,40,255},                 % 设定关键字颜色
	numberstyle=\footnotesize\color{darkgray},           
	commentstyle=\it\color[RGB]{0,96,96},                % 设置代码注释的格式
	stringstyle=\rmfamily\slshape\color[RGB]{128,0,0},   % 设置字符串格式
	showstringspaces=false,                              % 不显示字符串中的空格
	language=c,                                        % 设置语言
}




% 轻松引用, 可以用\cref{}指令直接引用, 自动加前缀. 
% 例: 图片label为fig:1
% \cref{fig:1} => Figure.1
% \ref{fig:1}  => 1
\usepackage[capitalize]{cleveref}
% \crefname{section}{Sec.}{Secs.}
\Crefname{section}{Section}{Sections}
\Crefname{table}{Table}{Tables}
\crefname{table}{Table.}{Tabs.}

%\setmainfont{Palatino Linotype.ttf}
%\setCJKmainfont{SimHei.ttf}
\setCJKsansfont{Songti.ttf}
% \setCJKmonofont{SimSun.ttf}
%\punctstyle{kaiming}
% 偏好的几个字体, 可以根据需要自行加入字体ttf文件并调用

\renewcommand{\emph}[1]{\begin{kaishu}#1\end{kaishu}}
\setlength{\parindent}{2em}
\everymath{\displaystyle}







\begin{document}
	\setlength{\baselineskip}{22pt}
	%若需在页眉部分加入内容, 可以在这里输入
	 \pagestyle{fancy}
	 \lhead{2023-2024学年秋季学期}
	 \chead{B2011003Y\quad 数字电路}
	 \rhead{实验说明文档}
	\begin{center}
		\Huge Alinx AX7035t开发板实验\\
		\Large ——基于埃氏筛法的6位素数计算与显示
	\end{center}
	\begin{center}
		\large \kaishu 姜俊彦\footnote{姜俊彦,中国科学院大学,北京,本次负责},刘镇豪\footnote{刘镇豪,中国科学院大学,北京,本次负责},吴尚哲\footnote{吴尚哲,中国科学院大学,北京,本次负责}
	\end{center}
	\tableofcontents
	\setcounter{page}{1}
	\thispagestyle{fancy}
	\newpage
	\section{实验题目}
	\large\kaishu  素数循环显示: 利用Alinx AX7035t开发板,使用Verilog编程实现计算并显示2-999999之间的素数。6个七段数码管显示素数,4个LED显示当前模式,4个按钮选择循环模式
	
	\begin{compactitem}
		\item 按键1:递增,每秒变一次(上电之后的默认模式)
		\item 按键2:递减,每秒变一次
		\item 按键3:递增,最快速度
		\item 按键4:递减,最快速度
	\end{compactitem}
	注意事项:
	\begin{compactenum}
		\item 不可以提前将素数计算好存储在FPGA内,必须运行时计算
		\item 每个小组独立完成,提交一份源码, 一份说明文档,均使用PDF电子版提交
		\item 组内成员最终共享大作业成绩,不做区别对待
	\end{compactenum}
	\section{实验过程}
	\songti
	\subsection{实验平台}
	\begin{compactitem}
		\item 操作系统:Windows 11 专业版 22H2
		\item 开发平台:Vivado v2023.1(64-bit)
		\item 硬件平台:XLINX ARTIX-7 系列 AX7035 FPGA 开发平台
		\item 协作平台:GitHub
	\end{compactitem}
	\subsection{理论调研}
	\subsection{总体规划}
	\subsection{开发日志}
	\newpage
	\section{实验代码}
	\newpage
	\subsection{}
	
	\subsection{}
	\section{实验难点}
	
	\section{功能介绍}
	
	\section{反思总结}
	
	
	
	\newpage
	\section{附录}
	\subsection{实验源码}
		\documentclass[11pt]{article}
\usepackage[a4paper]{geometry}
\geometry{left=2.0cm,right=2.0cm,top=2.5cm,bottom=2.5cm}
\usepackage{ctex} % 支持中文的LaTeX宏包
\usepackage{amsmath,amsfonts,graphicx,subfigure,amssymb,bm,amsthm,mathrsfs,mathtools,breqn} % 数学公式和符号的宏包集合
\usepackage{algorithm,algorithmicx} % 算法和伪代码的宏包
\usepackage[noend]{algpseudocode} % 算法和伪代码的宏包
\usepackage{fancyhdr} % 自定义页眉页脚的宏包
\usepackage[framemethod=TikZ]{mdframed} % 创建带边框的框架的宏包
\usepackage{fontspec} % 字体设置的宏包
\usepackage{adjustbox} % 调整盒子大小的宏包
\usepackage{fontsize} % 设置字体大小的宏包ddddddddd
\usepackage{tikz,xcolor} % 绘制图形和使用颜色的宏包
\usepackage{multicol} % 多栏排版的宏包
\usepackage{multirow} % 表格中合并单元格的宏包
\usepackage{pdfpages} % 插入PDF文件的宏包
\RequirePackage{listings} % 在文档中插入源代码的宏包
\RequirePackage{xcolor} % 定义和使用颜色的宏包
\usepackage{wrapfig} % 文字绕排图片的宏包
\usepackage{bigstrut,multirow,rotating} % 支持在表格中使用特殊命令的宏包
\usepackage{booktabs} % 创建美观的表格的宏包
\usepackage{circuitikz} % 绘制电路图的宏包
\usepackage{float} %
\usepackage{multirow}
\usepackage{indentfirst}
\usepackage{titlesec} %section宏
\usepackage{listings}
\usepackage{paralist}%压缩列表和行间列表环境
\usepackage[perpage]{footmisc}
\usepackage[colorlinks=false]{hyperref}
\usepackage{setspace}
%\usepackage[subfigure]{tocloft}      %必须这么写,否则会报错
%\renewcommand{\cftchapleader}{\cftdotfill{0.6}} %设置chapter条目的引导点间距
%\renewcommand{\cftsecleader}{\cftdotfill{0.6}}
%\renewcommand{\cftsubsecleader}{\cftdotfill{0.6}}
%\renewcommand{\cftchapfont}{\hts}    %设置chapter条目的字体
%\renewcommand{\cftsecfont}{\stxs}    %设置section条目的字体
%\renewcommand{\cftsubsecfont}{\stxs} %设置subsection条目的字体





\titleformat*{\section}{\LARGE\bfseries\songti}
\titleformat*{\subsection}{\Large\songti\bfseries}
\titleformat*{\subsubsection}{\large\songti\bfseries}
\titleformat*{\paragraph}{\large\songti\bfseries}
\titleformat*{\subparagraph}{\large\songti\bfseries}



\definecolor{dkgreen}{rgb}{0,0.6,0}
\definecolor{gray}{rgb}{0.5,0.5,0.5}
\definecolor{mauve}{rgb}{0.58,0,0.82}
\lstset{
	frame=tb,
	aboveskip=3mm,
	belowskip=3mm,
	showstringspaces=false,
	columns=flexible,
	framerule=1pt,
	rulecolor=\color{gray!35},
	%backgroundcolor=\color{gray\Delta 5},
	basicstyle={\small\ttfamily},
	numbers=none,
	numberstyle=\tiny\color{gray},
	keywordstyle=\color{blue},
	commentstyle=\color{dkgreen},
	stringstyle=\color{mauve},
	breaklines=true,
	breakatwhitespace=true,
	tabsize=3,
}


\lstdefinestyle{verilog}{
	columns=fixed,       
	numbers=left,                                        % 在左侧显示行号
	numberstyle=\tiny\color{gray},                       % 设定行号格式
	frame=single,                                          % 不显示背景边框
	backgroundcolor=\color[RGB]{255,255,255},            % 设定背景颜色
	keywordstyle=\color[RGB]{40,40,255},                 % 设定关键字颜色
	numberstyle=\footnotesize\color{darkgray},           
	commentstyle=\it\color[RGB]{0,96,96},                % 设置代码注释的格式
	stringstyle=\rmfamily\slshape\color[RGB]{128,0,0},   % 设置字符串格式
	showstringspaces=false,                              % 不显示字符串中的空格
	language=verilog,                                        % 设置语言
}

\lstdefinestyle{c}{
	columns=fixed,       
	numbers=left,                                        % 在左侧显示行号
	numberstyle=\tiny\color{gray},                       % 设定行号格式
	frame=single,                                          % 不显示背景边框
	backgroundcolor=\color[RGB]{255,255,255},            % 设定背景颜色
	keywordstyle=\color[RGB]{40,40,255},                 % 设定关键字颜色
	numberstyle=\footnotesize\color{darkgray},           
	commentstyle=\it\color[RGB]{0,96,96},                % 设置代码注释的格式
	stringstyle=\rmfamily\slshape\color[RGB]{128,0,0},   % 设置字符串格式
	showstringspaces=false,                              % 不显示字符串中的空格
	language=c,                                        % 设置语言
}




% 轻松引用, 可以用\cref{}指令直接引用, 自动加前缀. 
% 例: 图片label为fig:1
% \cref{fig:1} => Figure.1
% \ref{fig:1}  => 1
\usepackage[capitalize]{cleveref}
% \crefname{section}{Sec.}{Secs.}
\Crefname{section}{Section}{Sections}
\Crefname{table}{Table}{Tables}
\crefname{table}{Table.}{Tabs.}

%\setmainfont{Palatino Linotype.ttf}
%\setCJKmainfont{SimHei.ttf}
\setCJKsansfont{Songti.ttf}
% \setCJKmonofont{SimSun.ttf}
%\punctstyle{kaiming}
% 偏好的几个字体, 可以根据需要自行加入字体ttf文件并调用

\renewcommand{\emph}[1]{\begin{kaishu}#1\end{kaishu}}
\setlength{\parindent}{2em}
\everymath{\displaystyle}







\ifx\allfiles\undefined
\begin{document}
\else
\fi
\setlength{\baselineskip}{22pt}
%若需在页眉部分加入内容, 可以在这里输入
\pagestyle{fancy}
\lhead{2023-2024学年秋季学期}
\chead{B2011003Y\quad 数字电路}
\rhead{实验源码}
\begin{center}
	\Huge Alinx AX7035t开发板实验\\
	\Large ——基于埃氏筛法的6位素数计算与显示
\end{center}
	\begin{lstlisting}[style=verilog]
	/////////////////////////////////////////////////////////////////////////////////////
	module top
	(
		input           clk,
		input           rstn,
		output [3:0]    led,
		input  [3:0]    key,
		output [5:0]    seg_sel,
		output [7:0]    seg_dig
	);
	wire [3:0]          key_signal;
	wire [3:0]          key_pulse;
	wire                rstn_signal;
	wire                tick;
	wire                one_second;
	wire                select;
	wire                reset;
	wire [47:0]         seg;
	wire [19:0]         cnt_20b;
	wire [23:0]         cnt_24d;
	
	//1秒计时器
	Count_to_one_second timer(clk,one_second);
	
	//按键除抖及脉冲
	Killshake Killshake(clk,rstn,rstn_signal);
	genvar j;
	generate for(j = 0; j < 4; j = j + 1) begin
		Killshake Killshake (clk,key[j],key_signal[j]);
		Edgedetect Edgedetect (key_signal[j],clk,key_pulse[j]);
	end
	endgenerate
	
	//Led控制及模式选择
	ledcontrol ledcontrol(clk,rstn_signal,key_pulse,led);
	modecontrol modecontrol(clk,rstn_signal,one_second,led,key_pulse,reset,select,tick);
	
	//埃氏筛法
	isprime solver(clk,reset,tick,select,cnt_20b);
	
	//显示模块
	binary_20b_to_bcd_6d transformer(cnt_20b,cnt_24d);
	genvar i;
	generate for(i=0; i<6; i=i+1) begin
		led7seg_decode d(cnt_24d[i*4 +: 4], 1'b1, seg[i*8 +: 8]);
	end
	endgenerate
	seg_driver #(6) driver(clk, rstn_signal, 6'b111111, seg, seg_sel, seg_dig);
	
	endmodule
	/////////////////////////////////////////////////////////////////////////////////////
	module ledcontrol
	(
		input           clk,
		input           rstn_signal,
		input  [3:0]    key_pulse,
		output [3:0]    led
	);
	
	reg    [3:0]    led_r;
	always @(posedge clk or negedge rstn_signal) begin
		if(~rstn_signal) begin
			led_r <= 4'b1110;
		end
		else if(~key_pulse[0]) begin
			led_r <= 4'b1110;
		end
		else if(~key_pulse[1]) begin
			led_r <= 4'b1101;
		end
		else if(~key_pulse[2]) begin
			led_r <= 4'b1011;
		end
		else if(~key_pulse[3]) begin
			led_r <= 4'b0111;
		end
	end    
	assign led = led_r;
	endmodule
	/////////////////////////////////////////////////////////////////////////////////////
	module modecontrol
	(
		input           clk,
		input           rstn_signal,
		input           one_second,
		input [3:0]     led,
		input [3:0]     key_pulse,
		output          reset,
		output          select,
		output          tick
	);
	
	reg                 tick_r;
	reg                 select_reg;
	reg  [1:0]          reset_r;
	
	always @(posedge clk) begin
		if(~led[0]) begin
			tick_r <= one_second;
			select_reg<=1;
		end
		else if(~led[1])begin
			tick_r <= one_second;
			select_reg<=0;
		end
		else if(~led[2])begin
			tick_r <= 1;
			select_reg<=1;
		end
		else if(~led[3])begin
			tick_r <= 1;
		select_reg<=0;
		end
		else begin
			tick_r<=0;
		end
	end
	
	always @(posedge clk) begin
		reset_r<={reset_r[0],(&key_pulse)&rstn_signal};
	end
	
	assign tick = tick_r;
	assign select=select_reg;
	assign reset=reset_r[1];

	endmodule
	/////////////////////////////////////////////////////////////////////////////////////
	module Edgedetect
	(
		input   key,   
		input   clk,    
		output  pulse  
	);
	
	reg     key_prev;   
	reg     pulse_reg;  
	
	always @(posedge clk) begin
		key_prev <= key;
		if (key_prev & ~key) 
			pulse_reg <= 0;
		else 
			pulse_reg <= 1;
		end
	
	assign pulse = pulse_reg;
	
	endmodule
	/////////////////////////////////////////////////////////////////////////////////////
	module Killshake
	(
	input   clk,    
	input   key,    
	output  signal  
	);
	
	parameter   DEBOUNCE_TIME = 1000000;    
	reg [19:0]  count;                      
	reg         key_state;                  
	reg         signal_reg;                
	
	always @(posedge clk) begin
		if (key == key_state) begin
		if (count < DEBOUNCE_TIME) 
			count <= count + 1;
		else
			signal_reg <= key_state; 
		end 
		else begin
			count <= 0;
			key_state <= key;
		end
	end
	
	assign signal = signal_reg;
	
	endmodule
	/////////////////////////////////////////////////////////////////////////////////////
	module seg_driver #(parameter NPorts=8)
	(
	input                     clk, rstn, 
	input [NPorts-1:0]        valid_i, 
	input [NPorts*8-1:0]      seg_i, 
	output reg [NPorts-1:0]   valid_o, 
	output [7:0]              seg_o 
	);
	
	reg [14:0]          cnt;       
	reg [NPorts-1:0]    sel;       
	always @(posedge clk or negedge rstn) 
		if(~rstn) 
			cnt <= 0;
		else
			cnt <= cnt + 1;
	
	always @(posedge clk or negedge rstn) 
		if(~rstn)
			sel <= 0;
		else if(cnt == 0)
			sel <= (sel == NPorts - 1) ? 0 : sel + 1; 
	
	always @(sel, valid_i) begin 
		valid_o = {NPorts{1'b1}}; 
		valid_o[sel] = ~valid_i[sel]; 
	end
	
	assign seg_o = ~seg_i[sel*8+:8]; 
	
	endmodule
	/////////////////////////////////////////////////////////////////////////////////////
	module led7seg_decode
	(
		input [3:0] digit,
		input valid,
		output reg [7:0] seg
	);
	always @(digit)
		if(valid)
		case(digit)
			0: seg = 8'b00111111;//0
			1: seg = 8'b00000110;//1
			2: seg = 8'b01011011;//2
			3: seg = 8'b01001111;//3
			4: seg = 8'b01100110;//4
			5: seg = 8'b01101101;//5
			6: seg = 8'b01111101;//6
			7: seg = 8'b00000111;//7
			8: seg = 8'b01111111;//8
			9: seg = 8'b01101111;//9
			default: seg = 0;
		endcase
		else seg = 8'd0;
	endmodule
	/////////////////////////////////////////////////////////////////////////////////////
	module isprime #(parameter N=999999)
	(
		input clk,rstn,tick,
		input select,
		output [19:0] cnt_20b
	);
	reg [19:0]      w_addr;	       
	reg             w_data;	       
	reg             wea;	        
	reg [19:0]      r_addr;       
	wire            r_data;	        
	
	reg [19:0]      i;           
	reg [19:0]      j;         
	reg             en;
	reg             done;
	
	
	reg [19:0]      cnt_temp_reg;
	reg [19:0]      cnt_20b_reg;
	
	reg [2:0]       timer;
	reg             hold;
	
	always @(posedge clk or negedge rstn) begin
		if(!rstn) begin
			cnt_temp_reg<=(select)?2:N;
			cnt_20b_reg<=(select)?2:N;
			wea<=0;
			i<=2;
			j<=0;
			en<=0;
			done<=0;
			timer<=0;
			hold<=1;
		end
		else if (i*i<=N) begin
			if(en==0)begin
				r_addr<=i;
				if(timer>2)begin
					timer<=0;
					hold<=0;
				end
				else begin
					timer<=timer+1;
					hold<=1;
				end
				if(!hold) begin
					if (r_data==0) begin
						en<=1;
						j<=i+i;
					end
					else begin
						i<=i+1;
					end
				end
			end
			else if(en==1) begin
				if(j<N)begin
					wea<=1;
					w_addr<=j;
					w_data<=1;
					j<=j+i;
				end
				else begin
					wea<=0;
					en<=0;
					i<=i+1;
				end
			end 
		end
		else begin
			done<=1;
			if(done) begin
				if (((cnt_temp_reg<N)&(select))|((cnt_temp_reg>=2)&(~select))) begin
					r_addr<=cnt_temp_reg;
					if(hold) begin
						if(timer>2)begin
							timer<=0;
							hold<=0;
						end
						else begin
							timer<=timer+1;
						end
					end
					else begin
						if ((~r_data)&tick) begin
							cnt_20b_reg<=cnt_temp_reg;
							cnt_temp_reg<=(select)?cnt_temp_reg+1:cnt_temp_reg-1;
							hold<=1;
						end
						else if ((r_data)) begin
							cnt_temp_reg<=(select)?cnt_temp_reg+1:cnt_temp_reg-1;
							hold<=1;
						end
						else if((~r_data)&(~tick)) begin
							cnt_temp_reg<=cnt_temp_reg;
						end
					end
				end
			end
		end
	end
	assign cnt_20b=cnt_20b_reg;
	ram_ip ram_ip_inst_1 
	(
	.clka      (clk          ),     
	.wea       (wea          ),    
	.addra     (w_addr       ),    
	.dina      (w_data       ),     
	.clkb      (clk          ),     
	.addrb     (r_addr       ),     
	.doutb     (r_data       )     
	);
	endmodule 
	/////////////////////////////////////////////////////////////////////////////////////
	module binary_20b_to_bcd_6d #(parameter N = 20,parameter M = 24) 
	(    
		input  [N-1:0]  input_20b,       
		output [M-1:0]  output_6d       
	);  
	reg [3:0]       digits [5:0];       
	
	integer i;    
	always @(input_20b) begin     
		for (i = 0; i < 6; i = i+1) begin
			digits[i] = 4'd0;  
		end 
		for(i = N-1; i >= 0; i = i-1) begin  //加3移位法 
			if (digits[0] >= 4'b0101) digits[0] = digits[0] + 4'b0011;   
			if (digits[1] >= 4'b0101) digits[1] = digits[1] + 4'b0011;   
			if (digits[2] >= 4'b0101) digits[2] = digits[2] + 4'b0011;   
			if (digits[3] >= 4'b0101) digits[3] = digits[3] + 4'b0011;     
			if (digits[4] >= 4'b0101) digits[4] = digits[4] + 4'b0011;    
			if (digits[5] >= 4'b0101) digits[5] = digits[5] + 4'b0011;    
					
			digits[5][3:0] = {digits[5][2:0], digits[4][3]};  
			digits[4][3:0] = {digits[4][2:0], digits[3][3]};  
			digits[3][3:0] = {digits[3][2:0], digits[2][3]};    
			digits[2][3:0] = {digits[2][2:0], digits[1][3]};  
			digits[1][3:0] = {digits[1][2:0], digits[0][3]};   
			digits[0][3:0] = {digits[0][2:0], input_20b[i]};   
		end  
	end	   
	 
	assign output_6d ={digits[5],digits[4],digits[3],digits[2],digits[1],digits[0]};   
	
	endmodule 
	/////////////////////////////////////////////////////////////////////////////////////
	module Count_to_one_second #(parameter Count_To = 50_000_000)
	(
		input clk,
		output one_second
	);
	
	reg [31:0]  counter;
	reg         one_second_r;
	
	always @(posedge clk) begin
		if((counter < Count_To) && ~one_second_r) begin
			counter <= counter + 1;
		end
		else if((counter < Count_To) && one_second_r) begin
			one_second_r <= 0;
			counter <= counter + 1;
		end
		else begin
			counter <= 0;
			one_second_r <= 1;
		end
	end
	
	assign one_second = one_second_r;
	
	endmodule
	\end{lstlisting}
	
\ifx\allfiles\undefined
\end{document}
\else
\fi


		\newpage
	\subsection{开发日志}
		%%\documentclass[11pt]{article}
\usepackage[a4paper]{geometry}
\geometry{left=2.0cm,right=2.0cm,top=2.5cm,bottom=2.5cm}
\usepackage{ctex} % 支持中文的LaTeX宏包
\usepackage{amsmath,amsfonts,graphicx,subfigure,amssymb,bm,amsthm,mathrsfs,mathtools,breqn} % 数学公式和符号的宏包集合
\usepackage{algorithm,algorithmicx} % 算法和伪代码的宏包
\usepackage[noend]{algpseudocode} % 算法和伪代码的宏包
\usepackage{fancyhdr} % 自定义页眉页脚的宏包
\usepackage[framemethod=TikZ]{mdframed} % 创建带边框的框架的宏包
\usepackage{fontspec} % 字体设置的宏包
\usepackage{adjustbox} % 调整盒子大小的宏包
\usepackage{fontsize} % 设置字体大小的宏包ddddddddd
\usepackage{tikz,xcolor} % 绘制图形和使用颜色的宏包
\usepackage{multicol} % 多栏排版的宏包
\usepackage{multirow} % 表格中合并单元格的宏包
\usepackage{pdfpages} % 插入PDF文件的宏包
\RequirePackage{listings} % 在文档中插入源代码的宏包
\RequirePackage{xcolor} % 定义和使用颜色的宏包
\usepackage{wrapfig} % 文字绕排图片的宏包
\usepackage{bigstrut,multirow,rotating} % 支持在表格中使用特殊命令的宏包
\usepackage{booktabs} % 创建美观的表格的宏包
\usepackage{circuitikz} % 绘制电路图的宏包
\usepackage{float} %
\usepackage{multirow}
\usepackage{indentfirst}
\usepackage{titlesec} %section宏
\usepackage{listings}
\usepackage{paralist}%压缩列表和行间列表环境
\usepackage[perpage]{footmisc}
\usepackage[colorlinks=false]{hyperref}
\usepackage{setspace}
%\usepackage[subfigure]{tocloft}      %必须这么写,否则会报错
%\renewcommand{\cftchapleader}{\cftdotfill{0.6}} %设置chapter条目的引导点间距
%\renewcommand{\cftsecleader}{\cftdotfill{0.6}}
%\renewcommand{\cftsubsecleader}{\cftdotfill{0.6}}
%\renewcommand{\cftchapfont}{\hts}    %设置chapter条目的字体
%\renewcommand{\cftsecfont}{\stxs}    %设置section条目的字体
%\renewcommand{\cftsubsecfont}{\stxs} %设置subsection条目的字体





\titleformat*{\section}{\LARGE\bfseries\songti}
\titleformat*{\subsection}{\Large\songti\bfseries}
\titleformat*{\subsubsection}{\large\songti\bfseries}
\titleformat*{\paragraph}{\large\songti\bfseries}
\titleformat*{\subparagraph}{\large\songti\bfseries}



\definecolor{dkgreen}{rgb}{0,0.6,0}
\definecolor{gray}{rgb}{0.5,0.5,0.5}
\definecolor{mauve}{rgb}{0.58,0,0.82}
\lstset{
	frame=tb,
	aboveskip=3mm,
	belowskip=3mm,
	showstringspaces=false,
	columns=flexible,
	framerule=1pt,
	rulecolor=\color{gray!35},
	%backgroundcolor=\color{gray\Delta 5},
	basicstyle={\small\ttfamily},
	numbers=none,
	numberstyle=\tiny\color{gray},
	keywordstyle=\color{blue},
	commentstyle=\color{dkgreen},
	stringstyle=\color{mauve},
	breaklines=true,
	breakatwhitespace=true,
	tabsize=3,
}


\lstdefinestyle{verilog}{
	columns=fixed,       
	numbers=left,                                        % 在左侧显示行号
	numberstyle=\tiny\color{gray},                       % 设定行号格式
	frame=single,                                          % 不显示背景边框
	backgroundcolor=\color[RGB]{255,255,255},            % 设定背景颜色
	keywordstyle=\color[RGB]{40,40,255},                 % 设定关键字颜色
	numberstyle=\footnotesize\color{darkgray},           
	commentstyle=\it\color[RGB]{0,96,96},                % 设置代码注释的格式
	stringstyle=\rmfamily\slshape\color[RGB]{128,0,0},   % 设置字符串格式
	showstringspaces=false,                              % 不显示字符串中的空格
	language=verilog,                                        % 设置语言
}

\lstdefinestyle{c}{
	columns=fixed,       
	numbers=left,                                        % 在左侧显示行号
	numberstyle=\tiny\color{gray},                       % 设定行号格式
	frame=single,                                          % 不显示背景边框
	backgroundcolor=\color[RGB]{255,255,255},            % 设定背景颜色
	keywordstyle=\color[RGB]{40,40,255},                 % 设定关键字颜色
	numberstyle=\footnotesize\color{darkgray},           
	commentstyle=\it\color[RGB]{0,96,96},                % 设置代码注释的格式
	stringstyle=\rmfamily\slshape\color[RGB]{128,0,0},   % 设置字符串格式
	showstringspaces=false,                              % 不显示字符串中的空格
	language=c,                                        % 设置语言
}




% 轻松引用, 可以用\cref{}指令直接引用, 自动加前缀. 
% 例: 图片label为fig:1
% \cref{fig:1} => Figure.1
% \ref{fig:1}  => 1
\usepackage[capitalize]{cleveref}
% \crefname{section}{Sec.}{Secs.}
\Crefname{section}{Section}{Sections}
\Crefname{table}{Table}{Tables}
\crefname{table}{Table.}{Tabs.}

%\setmainfont{Palatino Linotype.ttf}
%\setCJKmainfont{SimHei.ttf}
\setCJKsansfont{Songti.ttf}
% \setCJKmonofont{SimSun.ttf}
%\punctstyle{kaiming}
% 偏好的几个字体, 可以根据需要自行加入字体ttf文件并调用

\renewcommand{\emph}[1]{\begin{kaishu}#1\end{kaishu}}
\setlength{\parindent}{2em}
\everymath{\displaystyle}








\ifx\allfiles\undefined
\begin{document}
\else
\fi


\begin{compactitem}
	\item 2023-11-23:
	\begin{compactitem}
		\item 建立Github协作库、添加模板文件top.sv
		\item 拿到AX7035 FPGA 开发板
	\end{compactitem}
	\item 2023-11-24:
	\begin{compactitem}
		\item Github协作邀请完毕
	\end{compactitem}
	\item 2023-12-07:
	\begin{compactitem}
		\item 总体布局与任务分发
		\begin{compactitem}
			\item Edgedetect.v-刘镇豪
			\item KillShake.v-刘镇豪
			\item FIFO.v-吴尚哲
			\item LED\_display.v-吴尚哲
			\item binary\_20b\_to\_bcd\_6d.v-规划中
		\end{compactitem}
		\item SRAM片上内存
		\begin{compactitem}
			\item 完成了.xdc管脚协议的补充(后证实不需要)
			\item 研究了SRAM的结构与原理
		\end{compactitem}
		\item 库文件细化
		\begin{compactitem}
			\item 建立了参考文献集Reference.txt
			\item 建立了重要信息共享文档ShareLog.md
			\item 建立了样本数据集datadic.txt
		\end{compactitem}
\end{compactitem}
	\item 2023-12-09 
	\begin{compactitem}
		\item 关于AX7035开发板
		\begin{compactitem}
			\item 找到了一份完备的教程
		\end{compactitem}
		\item 关于DDR3
		\begin{compactitem}
			\item 建立了DDR3的功能及驱动模块
			\item 建立了mem\_burst.v的读写模块,但是还未来得及分析
		\end{compactitem}
		\item 关于.xdc文件
		\begin{compactitem}
			\item 恢复了原.xdc样式,并对修改做了备份
		\end{compactitem}
		\item 关于top.sv
		\begin{compactitem}
			\item 仿照样例撰写了led7seg\_decode.v,本质为0-9二进制数到8端数码管数据译码器
			\item 写了一些注释:其中下面一段代码存疑
			\begin{lstlisting}[style=verilog]
genvar i;
generate 
	for(i=0; i<6; i=i+1) begin
		led7seg_decode d(cnt[i*4 +: 4], 1'b1, seg[i*8 +: 8]);//+是做什么的?
end
endgenerate
			\end{lstlisting}
		\end{compactitem}
		\item 关于组员
		\begin{compactitem}
			\item FIFO.v已完成
			\item LED\_display.v已完成
			\item KillShake.v已完成
			\item Edgedetect.v已完成
		\end{compactitem}
	\end{compactitem}
	\item 2023-12-17
	\begin{compactitem}
		\item 关于top.sv
		\begin{compactitem}
			\item 实现了防抖电路和脉冲输出的测试
			\item 撰写了指示灯显示与状态切换代码
		\end{compactitem}
	\end{compactitem}
	\item 2023-12-18
	\begin{compactitem}
		\item 关于top.sv
		\begin{compactitem}
			\item 实现了按键与`LED`灯对应的代码与测试-刘镇豪
			\item 探索了筛法的可能性并决定算法为埃氏筛法,初步完成了埃氏筛法的代码实现,未测试
		\end{compactitem}
		\item 总体任务分发
		\begin{compactitem}
			\item binary\_20b\_to\_bcd\_6d.v-吴尚哲
			\item Count\_to\_one\_second.v-刘镇豪
		\end{compactitem}
		\item 关于组员完成情况
		\begin{compactitem}
			\item Count\_to\_one\_second.v已完成,未测试
		\end{compactitem}
	\end{compactitem}
	\item 2023-12-23
	\begin{compactitem}
		\item 关于top.sv
		\begin{compactitem}
			\item 实现了八段数码管显示的代码编写及测试(10进制)
			\item 实现了一秒计时器的整合与编写
			\item 实现了埃氏筛法(算法层面),但是其对于内存地址的调用目前仍然存在问题
		\end{compactitem}
		\item 关于组员
		\begin{compactitem}
			\item binary\_20b\_to\_bcd\_6d.v已完成,已测试
			\item Count\_to\_one\_second.v已测试
		\end{compactitem}
	\end{compactitem}
	\item 2023-12-24
	\begin{compactitem}
		\item 关于top.sv
		\begin{compactitem}
			\item 实现了埃氏筛法,最快输出达到1s之内完成
		\end{compactitem}
	\end{compactitem}
	\item 2023-12-25
	\begin{compactitem}
		\item 关于top.sv
		\begin{compactitem}
			\item 实现了最快输出的递增和递减功能按钮对应,但是对于1s输出的复位目前仍存在问题
		\end{compactitem}
		\item 关于组员
		\begin{compactitem}
			\item 布置了实验报告撰写的相关任务
		\end{compactitem}
	\end{compactitem}
	\item 2023-12-26
	\begin{compactitem}
		\item 关于top.sv
		\begin{compactitem}
			\item 实现了实验要求的所有功能
			\item 美化了整体代码布局
		\end{compactitem}
		\item 关于库文件
		\begin{compactitem}
			\item 将所有模块分装为.v文件存储在src文件夹下
			\item 将未用到的代码及内容存储在misc/Unused文件夹下
		\end{compactitem}
	\end{compactitem}
	\item 2023-12-28
	\begin{compactitem}
		\item 关于实验报告
		\begin{compactitem}
			\item 完成了实验报告撰写
		\end{compactitem}
		\item 关于算法性能
		\begin{compactitem}
			\item 通过统计时钟周期数估计了算法性能
		\end{compactitem}
	\end{compactitem}
\end{compactitem}






\ifx\allfiles\undefined
\end{document}
\else
\fi


		%\newpage
	\begin{thebibliography}{3}
		\bibitem{ref1}
	\end{thebibliography}
	
\end{document}


