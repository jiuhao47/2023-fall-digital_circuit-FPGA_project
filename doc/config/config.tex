\documentclass[11pt]{article}
\usepackage[a4paper]{geometry}
\geometry{left=2.0cm,right=2.0cm,top=2.5cm,bottom=2.5cm}
\usepackage{ctex} % 支持中文的LaTeX宏包
\usepackage{amsmath,amsfonts,graphicx,subfigure,amssymb,bm,amsthm,mathrsfs,mathtools,breqn} % 数学公式和符号的宏包集合
\usepackage{algorithm,algorithmicx} % 算法和伪代码的宏包
\usepackage[noend]{algpseudocode} % 算法和伪代码的宏包
\usepackage{fancyhdr} % 自定义页眉页脚的宏包
\usepackage[framemethod=TikZ]{mdframed} % 创建带边框的框架的宏包
\usepackage{fontspec} % 字体设置的宏包
\usepackage{adjustbox} % 调整盒子大小的宏包
\usepackage{fontsize} % 设置字体大小的宏包ddddddddd
\usepackage{tikz,xcolor} % 绘制图形和使用颜色的宏包
\usepackage{multicol} % 多栏排版的宏包
\usepackage{multirow} % 表格中合并单元格的宏包
\usepackage{pdfpages} % 插入PDF文件的宏包
\RequirePackage{listings} % 在文档中插入源代码的宏包
\RequirePackage{xcolor} % 定义和使用颜色的宏包
\usepackage{wrapfig} % 文字绕排图片的宏包
\usepackage{bigstrut,multirow,rotating} % 支持在表格中使用特殊命令的宏包
\usepackage{booktabs} % 创建美观的表格的宏包
\usepackage{circuitikz} % 绘制电路图的宏包
\usepackage{float} %
\usepackage{multirow}
\usepackage{indentfirst}
\usepackage{titlesec} %section宏
\usepackage{listings}
\usepackage{paralist}%压缩列表和行间列表环境
\usepackage[colorlinks=false]{hyperref}

\titleformat*{\section}{\LARGE\bfseries\songti}
\titleformat*{\subsection}{\Large\songti\bfseries}
\titleformat*{\subsubsection}{\large\songti\bfseries}
\titleformat*{\paragraph}{\large\songti\bfseries}
\titleformat*{\subparagraph}{\large\songti\bfseries}



\definecolor{dkgreen}{rgb}{0,0.6,0}
\definecolor{gray}{rgb}{0.5,0.5,0.5}
\definecolor{mauve}{rgb}{0.58,0,0.82}
\lstset{
	frame=tb,
	aboveskip=3mm,
	belowskip=3mm,
	showstringspaces=false,
	columns=flexible,
	framerule=1pt,
	rulecolor=\color{gray!35},
	%backgroundcolor=\color{gray\Delta 5},
	basicstyle={\small\ttfamily},
	numbers=none,
	numberstyle=\tiny\color{gray},
	keywordstyle=\color{blue},
	commentstyle=\color{dkgreen},
	stringstyle=\color{mauve},
	breaklines=true,
	breakatwhitespace=true,
	tabsize=3,
}


\lstset{
	columns=fixed,       
	numbers=left,                                        % 在左侧显示行号
	numberstyle=\tiny\color{gray},                       % 设定行号格式
	frame=none,                                          % 不显示背景边框
	backgroundcolor=\color[RGB]{245,245,244},            % 设定背景颜色
	keywordstyle=\color[RGB]{40,40,255},                 % 设定关键字颜色
	numberstyle=\footnotesize\color{darkgray},           
	commentstyle=\it\color[RGB]{0,96,96},                % 设置代码注释的格式
	stringstyle=\rmfamily\slshape\color[RGB]{128,0,0},   % 设置字符串格式
	showstringspaces=false,                              % 不显示字符串中的空格
	language=verilog,                                        % 设置语言
}




% 轻松引用, 可以用\cref{}指令直接引用, 自动加前缀. 
% 例: 图片label为fig:1
% \cref{fig:1} => Figure.1
% \ref{fig:1}  => 1
\usepackage[capitalize]{cleveref}
% \crefname{section}{Sec.}{Secs.}
\Crefname{section}{Section}{Sections}
\Crefname{table}{Table}{Tables}
\crefname{table}{Table.}{Tabs.}

%\setmainfont{Palatino Linotype.ttf}
%\setCJKmainfont{SimHei.ttf}
\setCJKsansfont{Songti.ttf}
% \setCJKmonofont{SimSun.ttf}
%\punctstyle{kaiming}
% 偏好的几个字体, 可以根据需要自行加入字体ttf文件并调用

\renewcommand{\emph}[1]{\begin{kaishu}#1\end{kaishu}}
\setlength{\parindent}{2em}
\everymath{\displaystyle}






