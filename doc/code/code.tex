\ifx\allfiles\undefined
\begin{document}
\else
\fi
	\begin{lstlisting}[style=verilog]
	module top
	(
	input           clk,
	input           rstn,
	output [3:0]    led,
	input  [3:0]    key,
	output [5:0]    seg_sel,
	output [7:0]    seg_dig
	);
	wire [3:0]          key_signal;
	wire [3:0]          key_pulse;
	wire                rstn_signal;
	wire                tick;
	wire                one_second;
	wire                select;
	wire                reset;
	wire [47:0]         seg;
	wire [19:0]         cnt_20b;
	wire [23:0]         cnt_24d;
	
	//1秒计时器
	Count_to_one_second timer(clk,one_second);
	
	//按键除抖及脉冲
	Killshake Killshake(clk,rstn,rstn_signal);
	genvar j;
	generate for(j = 0; j < 4; j = j + 1) begin
	Killshake Killshake (clk,key[j],key_signal[j]);
	Edgedetect Edgedetect (key_signal[j],clk,key_pulse[j]);
	end
	endgenerate
	
	//Led控制及模式选择
	ledcontrol ledcontrol(clk,rstn_signal,key_pulse,led);
	modecontrol modecontrol(clk,rstn_signal,one_second,led,key_pulse,reset,select,tick);
	
	//埃氏筛法
	isprime solver(clk,reset,tick,select,cnt_20b);
	
	//显示模块
	binary_20b_to_bcd_6d transformer(cnt_20b,cnt_24d);
	genvar i;
	generate for(i=0; i<6; i=i+1) begin
	led7seg_decode d(cnt_24d[i*4 +: 4], 1'b1, seg[i*8 +: 8]);
	end
	endgenerate
	seg_driver #(6) driver(clk, rstn_signal, 6'b111111, seg, seg_sel, seg_dig);//数码管驱动,48宽(6*8)数据显示
	
	endmodule
	
	
	module ledcontrol
	(
	input           clk,
	input           rstn_signal,
	input  [3:0]    key_pulse,
	output [3:0]    led
	);
	
	reg    [3:0]    led_r;
	always @(posedge clk or negedge rstn_signal) begin
	if(~rstn_signal) begin
	led_r <= 4'b1110;
	end
	else if(~key_pulse[0]) begin
	led_r <= 4'b1110;
	end
	else if(~key_pulse[1]) begin
	led_r <= 4'b1101;
	end
	else if(~key_pulse[2]) begin
	led_r <= 4'b1011;
	end
	else if(~key_pulse[3]) begin
	led_r <= 4'b0111;
	end
	end    
	assign led = led_r;
	endmodule
	
	
	
	
	module modecontrol
	(
	input           clk,
	input           rstn_signal,
	input           one_second,
	input [3:0]     led,
	input [3:0]     key_pulse,
	output          reset,
	output          select,
	output          tick
	);
	
	reg                 tick_r;
	reg                 select_reg;
	reg  [1:0]          reset_r;
	
	always @(posedge clk) begin
	if(~led[0]) begin
	tick_r <= one_second;
	select_reg<=1;
	end
	else if(~led[1])begin
	tick_r <= one_second;
	select_reg<=0;
	end
	else if(~led[2])begin
	tick_r <= 1;
	select_reg<=1;
	end
	else if(~led[3])begin
	tick_r <= 1;
	select_reg<=0;
	end
	else begin
	tick_r<=0;
	end
	end
	
	
	always @(posedge clk) begin
	reset_r<={reset_r[0],(&key_pulse)&rstn_signal};
	end
	
	assign tick = tick_r;
	assign select=select_reg;
	assign reset=reset_r[1];
	
	endmodule
	
	module Edgedetect
	(
	input   key,    // 按钮输入
	input   clk,    // 时钟信号
	output  pulse   // 脉冲输出
	);
	
	reg     key_prev;   // 存储前一个按键状态
	reg     pulse_reg;  // always块中储存状态
	
	
	always @(posedge clk) begin
	key_prev <= key;
	// 当检测到按键的负沿时,生成脉冲
	if (key_prev & ~key) 
	pulse_reg <= 0;
	else 
	pulse_reg <= 1;
	end
	
	assign pulse = pulse_reg;
	
	endmodule
	
	module Killshake
	(
	input   clk,    // 时钟信号
	input   key,    // 含噪声的按键输入
	output  signal  // 清洁的按键输出
	);
	
	parameter   DEBOUNCE_TIME = 1000000;    // 去抖时间阈值,根据时钟频率调整,1/50s
	reg [19:0]  count;                      // 计数器,位宽取决于DEBOUNCE_TIME
	reg         key_state;                  // 存储稳定后的按键状态
	reg         signal_reg;                 // always块中储存状态
	
	always @(posedge clk) begin
	if (key == key_state) begin
	// 如果当前输入状态与去抖后的状态相同,则增加计数器
	if (count < DEBOUNCE_TIME) 
	count <= count + 1;
	else
	signal_reg <= key_state; // 更新输出状态
	end else begin
	// 如果输入状态改变,重置计数器并更新去抖后的状态
	count <= 0;
	key_state <= key;
	end
	end
	
	assign signal = signal_reg;
	
	endmodule
	
	
	//数码管驱动
	module seg_driver #(parameter NPorts=8)
	(
	input                     clk, rstn, 
	input [NPorts-1:0]        valid_i, 
	input [NPorts*8-1:0]      seg_i, 
	output reg [NPorts-1:0]   valid_o, 
	output [7:0]              seg_o 
	);
	
	reg [14:0]          cnt;        // 15 位寄存器 cnt,用于计数,0<=cnt<=2^15-1
	reg [NPorts-1:0]    sel;        // NPorts 位(即8位)的寄存器 sel,用于选择当前输入端口
	always @(posedge clk or negedge rstn) 
	if(~rstn) 
	cnt <= 0;
	else
	cnt <= cnt + 1;
	
	always @(posedge clk or negedge rstn) 
	if(~rstn)
	sel <= 0;
	else if(cnt == 0)
	sel <= (sel == NPorts - 1) ? 0 : sel + 1; // 若条件(sel == NPorts - 1)为真,将sel赋值为0,否则sel+1,循环刷新
	
	always @(sel, valid_i) begin // 使用 sel 和 valid_i 作为敏感信号的 always 块
	valid_o = {NPorts{1'b1}}; // 初始化 valid_o 为全 1 的向量,表示所有输出端口有效
	valid_o[sel] = ~valid_i[sel]; // 取反当前选择的输入端口的有效性,表示相应输出端口的有效性
	end
	
	assign seg_o = ~seg_i[sel*8+:8]; //取反从sel_i寄存器索引开始选择的8位数据段,赋值给 seg_o
	
	endmodule
	
	module led7seg_decode
	(
	input [3:0] digit,
	input valid,
	output reg [7:0] seg
	);
	always @(digit)
	if(valid)
	case(digit)
	0: seg = 8'b00111111;//0
	1: seg = 8'b00000110;//1
	2: seg = 8'b01011011;//2
	3: seg = 8'b01001111;//3
	4: seg = 8'b01100110;//4
	5: seg = 8'b01101101;//5
	6: seg = 8'b01111101;//6
	7: seg = 8'b00000111;//7
	8: seg = 8'b01111111;//8
	9: seg = 8'b01101111;//9
	default: seg = 0;
	endcase
	else seg = 8'd0;
	
	endmodule
	
	
	module isprime #(parameter N=999999)
	(
	input clk,rstn,tick,
	input select,
	output [19:0] cnt_20b
	);
	reg [19:0]      w_addr;	        //写入的数据的地址
	reg             w_data;	        //写入的数据
	reg             wea;	        //使能端
	reg [19:0]      r_addr;         //读取的数据的地址
	wire            r_data;	        //读取的数据
	
	reg [19:0]      i;              //外层循环变量
	reg [19:0]      j;              //内层循环变量
	reg             en;
	reg             done;
	
	
	reg [19:0]      cnt_temp_reg;
	reg [19:0]      cnt_20b_reg;
	
	reg [2:0]       timer;
	reg             hold;
	
	always @(posedge clk or negedge rstn) begin
	if(!rstn) begin
	cnt_temp_reg<=(select)?2:N;
	cnt_20b_reg<=(select)?2:N;
	wea<=0;
	i<=2;
	j<=0;
	en<=0;
	done<=0;
	timer<=0;
	hold<=1;
	end
	else if (i*i<=N) begin
	if(en==0)begin
	r_addr<=i;
	if(timer>2)begin
	timer<=0;
	hold<=0;
	end
	else begin
	timer<=timer+1;
	hold<=1;
	end
	if(!hold) begin
	if (r_data==0) begin
	en<=1;
	j<=i+i;
	end
	else begin
	i<=i+1;
	end
	end
	end
	else if(en==1) begin
	if(j<N)begin
	wea<=1;
	w_addr<=j;
	w_data<=1;
	j<=j+i;
	end
	else begin
	wea<=0;
	en<=0;
	i<=i+1;
	end
	end 
	end
	else begin
	done<=1;
	if(done) begin
	if (((cnt_temp_reg<N)&(select))|((cnt_temp_reg>=2)&(~select))) begin
	r_addr<=cnt_temp_reg;
	if(hold) begin
	if(timer>2)begin
	timer<=0;
	hold<=0;
	end
	else begin
	timer<=timer+1;
	end
	end
	else begin
	if ((~r_data)&tick) begin
	cnt_20b_reg<=cnt_temp_reg;
	cnt_temp_reg<=(select)?cnt_temp_reg+1:cnt_temp_reg-1;
	hold<=1;
	end
	else if ((r_data)) begin
	cnt_temp_reg<=(select)?cnt_temp_reg+1:cnt_temp_reg-1;
	hold<=1;
	end
	else if((~r_data)&(~tick)) begin
	cnt_temp_reg<=cnt_temp_reg;
	end
	end
	end
	end
	end
	end
	assign cnt_20b=cnt_20b_reg;
	ram_ip ram_ip_inst_1 
	(
	.clka      (clk          ),     // input clka
	.wea       (wea          ),     // input [0 : 0] wea
	.addra     (w_addr       ),     // input [19 : 0] addra
	.dina      (w_data       ),     // input [0 : 0] dina
	.clkb      (clk          ),     // input clkb
	.addrb     (r_addr       ),     // input [19 : 0] addrb
	.doutb     (r_data       )      // output [0 : 0] doutb
	);
	endmodule //isprime
	
	
	
	module binary_20b_to_bcd_6d #(parameter N = 20,parameter M = 24) 
	(    
	input  [N-1:0]  input_20b,       
	output [M-1:0]  output_6d       
	);  
	reg [3:0]       digits [5:0];       
	
	integer i;    
	always @(input_20b) begin     
	for (i = 0; i < 6; i = i+1) begin
	digits[i] = 4'd0;  
	end 
	for(i = N-1; i >= 0; i = i-1) begin  //加3移位法 
	if (digits[0] >= 4'b0101) digits[0] = digits[0] + 4'b0011;   
	if (digits[1] >= 4'b0101) digits[1] = digits[1] + 4'b0011;   
	if (digits[2] >= 4'b0101) digits[2] = digits[2] + 4'b0011;   
	if (digits[3] >= 4'b0101) digits[3] = digits[3] + 4'b0011;     
	if (digits[4] >= 4'b0101) digits[4] = digits[4] + 4'b0011;    
	if (digits[5] >= 4'b0101) digits[5] = digits[5] + 4'b0011;    
	
	digits[5][3:0] = {digits[5][2:0], digits[4][3]};  
	digits[4][3:0] = {digits[4][2:0], digits[3][3]};  
	digits[3][3:0] = {digits[3][2:0], digits[2][3]};    
	digits[2][3:0] = {digits[2][2:0], digits[1][3]};  
	digits[1][3:0] = {digits[1][2:0], digits[0][3]};   
	digits[0][3:0] = {digits[0][2:0], input_20b[i]};   
	end  
	end	    
	assign output_6d ={digits[5],digits[4],digits[3],digits[2],digits[1],digits[0]};   
	
	endmodule 
	
	module Count_to_one_second #(parameter Count_To = 50_000_000)
	(
	input clk,
	output one_second
	);
	
	reg [31:0]  counter;
	reg         one_second_r;
	
	always @(posedge clk) begin
	if((counter < Count_To) && ~one_second_r) begin
	counter <= counter + 1;
	end
	else if((counter < Count_To) && one_second_r) begin
	one_second_r <= 0;
	counter <= counter + 1;
	end
	else begin
	counter <= 0;
	one_second_r <= 1;
	end
	end
	
	assign one_second = one_second_r;
	
	endmodule
	\end{lstlisting}
	
\ifx\allfiles\undefined
\end{document}
\else
\fi

